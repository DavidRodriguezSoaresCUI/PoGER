% !TeX spellcheck = en_US
\documentclass[11pt]{article}

\usepackage{amssymb} % see: http://milde.users.sourceforge.net/LUCR/Math/mathpackages/amssymb-symbols.pdf
\usepackage{mathtools} % for extended mathematical symbols and more

% Page properties :
% See : https://tex.stackexchange.com/questions/36085/latex-without-pages
\usepackage{geometry}
\geometry{margin=16mm}

\usepackage[english]{babel}
\usepackage{lipsum}

\newcommand\BS{\char`\\}

% To color text :
% see : https://en.wikibooks.org/wiki/LaTeX/Colors
\usepackage[dvipsnames]{xcolor}

% Input file encoding :
\usepackage[utf8]{inputenc}     % makes accents not shit

% To use images : 
\usepackage{graphicx} % use this
\graphicspath{ {./IMG/} } % path midifier for storing images
% to include a picture, type this ('name' is without path or extention) :
% \includegraphics[width=0.5\textwidth]{name}

\usepackage{dirtytalk}
\usepackage{parskip}

% How to use references :
\usepackage{hyperref} % use this
\hypersetup{
	colorlinks=true,    
	urlcolor=blue,
}

\usepackage{wasysym}

\usepackage{color}

\definecolor{mylightgray}{rgb}{0.95,0.95,0.95}

\usepackage[super]{nth}

% Regulates \tableofcontents max depth
\setcounter{tocdepth}{2}

\usepackage{array}
\newcolumntype{L}[1]{>{\raggedright\let\newline\\\arraybackslash\hspace{0pt}}m{#1}}
\newcolumntype{C}[1]{>{\centering\let\newline\\\arraybackslash\hspace{0pt}}m{#1}}
\newcolumntype{R}[1]{>{\raggedleft\let\newline\\\arraybackslash\hspace{0pt}}m{#1}}
\usepackage{xcolor,colortbl}
\newcolumntype{a}{>{\columncolor{mylightgray}}l}

%% Define a HUGE : https://tex.stackexchange.com/questions/265/fonts-larger-than-huge
\usepackage{fix-cm}   
\makeatletter
\newcommand\HUGE{\@setfontsize\Huge{29}{35}}
\makeatother 

%% Adds padding to tables : https://tex.stackexchange.com/questions/64761/add-just-a-little-more-padding-to-my-table/381439
\usepackage{array}
\setlength\extrarowheight{2pt} % or whatever amount is appropriate

%opening
\title{\vspace{15mm}{\HUGE PoGER}\\
	A Bachelor's project}
\author{}
\date{}

% first page
\begin{document}
\maketitle

\vspace{90mm}

\textbf{Privacy policy}

This is a confidential document and should not be distributed under any circumstance. \hyperref[sec:privacypolice]{Please click and read}.

\textbf{Abstract}

\lipsum[2-3]


\newpage

\begingroup
\hypersetup{linkcolor=black}
\tableofcontents
\endgroup

\newpage
\section{Inception}

\subsection{A Bachelor's project's journey}

This is intended to document my journey through my Bachelor's project. Some information may be omitted for brevity or because it is present in accompanying scripts.

There are 3 major steps : inception, research and implementation.

\subsection{Brainsorm}

This is the \textbf{first step} of a Bachelor's project : choosing what to do. It took me significantly more time that I anticipated before I converged towards an answer.

When hunting for a project idea, I listed a few concepts that I would like them to feature :
\begin{itemize}
	\item Meta-programming
	\item Creating a domain specific language (descriptive or interpreted)
	\item Multi-platform app
	\item A game
	\item Something that would be useful to many people, developers or users
	\item An open-source project, free software if possible
\end{itemize}

Here were my requirements :
\begin{itemize}
	\item I didn't want to work on web-related stuff. That means no servlets, microservices, HTML or CSS, etc
	
	\item I wanted to work with few programming languages, my preference going to Python.
	
	\item I didn't want to just do a research work. I wanted to deliver an actual piece of software.
	
	\item The project should be some sort of program with an user interface.
\end{itemize}

I had a hard time trying to find actual ideas that both matched my requirements and made any sense as an academic work.


\subsection{Quest for an Original Idea}

In my quest for an original idea for a project, I discussed with friends and professors. Some professors had project proposition in their fields, none of which I found particularly attractive.

One of my friends worked on studying a piece of simulation software and its physics engine, which was inspiring but not quite what I was looking for.

I explored a few concepts that went nowhere, but eventually I settled on the following : I wanted to try studying a piece of software called \textbf{Pokemon Essentials}, a game engine of sorts, and the games made with it. Moreover, I wanted to do something with it, come up with an idea for a project that branched from it.

This is the beginning of the transition to the \textbf{research stage} of the project.

\newpage
\subsection{Small aside}

My concept choice need an explanation : I remembered a game I played years ago. It was a Pokemon game developped on RPG Maker XP. I was curious at the time and wondered if I could contribute in some way to this kind of projects. My curiosity led me to research how they were developed.

What I discovered was a clustered universe of fan-made games :
\begin{itemize}
	\item ROM hacks : games that build upon the code commercial products, typically GameBoy games.
	\begin{itemize}
		\item[+] They can be played on common emulatory on any platform.
		\item[-] Programming for them is very technical.
	\end{itemize}
	
	\item Games made with RPG Maker XP : They build on top of the \textit{Pokemon Essentials} project.
	
	\begin{itemize}
		\item[+] Easier to program for than ROM hacks, has less limitations and RPG Maker XP offers convenient tools.
		\item[-] RPG Maker XP is an aging RPG engine, programming for them is technical and they are only compatible with Windows (even running them on compatibility layers such as Wine on Linux isn't always functional). Also, performance is typically poor and RPG Maker XP is a closed-source software, making porting attempts extremely complex. Few projects were completed, typically closed-source.
	\end{itemize}

	\item Original games : Original software, typically built for Windows or the web.
	\begin{itemize}
		\item[+] Can have great performance, new game mechanics (eg. MMORPG style)
		\item[-] Typically closed-source, few projects exist.
	\end{itemize}
\end{itemize}

Quickly, I got discouraged by the apparent complexity and the sheer amount of failed/incomplete similar games (\href{https://pokemon-fan-game.fandom.com/wiki/Category:Incomplete_Games}{examples here}). But I never let go the idea of contributing that development scene.


\subsection{Defining the project through research}

During the research stage of the project, I experimented with RPG Maker XP, Pokemon Essentials and projects based on it. This was done to obtain an understanding of how they work and which direction to choose.

What I rapidly determined was that these tools weren't adapted to creating a game in a modern way, and suffered from their age and design decision, to the point that they were basically abandoned and deprecated. It was clear to me that I should not try to build upon them, but rather bring them up somehow.

I won't detail every design decision here, but each one was made toward the goal of addressing the identified shortcomings of the studied software. Please read the \textit{vision document} for more information.

At first I looked into implementing a similar engine that would be decoupled from RPG Maker XP  and would run games as programs written in an interpreted domain-specific language I would create. A complementary piece of software would automate the port of games written for the original Pokemon Essentials engine.

The concept of \href{https://en.wikipedia.org/wiki/Game_engine_recreation}{game engine recreation} seemed fitting for an academic work, so I validated that idea. 

As I looked more into it, the project concept got more refined. As a placeholder name, \textbf{PoGER} (\textit{\textbf{Po}kemon Essentials \textbf{G}ame \textbf{E}ngine \textbf{R}ecreation}) was chosen.

%TODO

The Problem Statement is a crucial : it defines the observed issues with the current situation and what the objectives for the proposed solution are. The following is taken from the \textit{vision document} for PoGER :


\begin{tabular}{|a|L{.66\linewidth}|}
	\hline
	The problem of & \begin{itemize}
		\item PE being based on RPGXP, therefore being bound to Windows
		
		\item Performance being poor
		
		\item PE being a convoluted solution to make fangames
		
		\item The need to being fluent in RPGXP-specific Ruby
		
		\item Having no separation between game code and PE code
		
		\item General lack of standards and documentation
		
		\item PE being fan-made and not well maintained
		
		\item Each fangame is isolated, and typically lacks multi-player or online features.
	\end{itemize} \\
	\hline
	affects & \begin{itemize}
		\item Anyone willing to code/coding a fan-game 
		\item People willing to play fangames on other platforms
	\end{itemize} \\
	\hline
	the impact of which is & a variety of unnecessary technical problems \\
	\hline
	A successful solution would be & An implementation of PE that is : \begin{itemize}
		\item performant
		\item available (multi-platform)
		\item straightforward and simpler to develop for
		\item allows to discover fangames
		\item has online capabilities
		\item is separate from the fangames
		\item provides documentation, sets standards
		\item open-source to allow contributors to maintain it
		\item has a way to port games from the original PE
	\end{itemize} \\
	\hline
\end{tabular}

As you can see, I believe there are sufficient issues with the current solution to warrant an alternative.

The following sections focus on documenting what the studied software are and some of the useful information I learned about them.


\subsubsection{Abbreviations}

Here are some of the abbreviations you can encounter :
\begin{itemize}
	\item RMXP - RPG Maker XP
	\item RGSS - Ruby Game Scripting System
	\item PE - Pokemon Essentials
	\item PU - Pokemon Uranium
\end{itemize}





\newpage

\section{Research subjects}
\subsection{RPG Maker XP}


According to its \href{https://en.wikipedia.org/wiki/RPG_Maker#RPG_Maker_series_timeline}{Wikipedia article} and its \href{https://www.rpgmakerweb.com/products/programs/rpg-maker-xp}{official website}, RPG Maker XP :
\begin{itemize}
	\item was developped by Enterbrain and released on July of 2004.
	
	\item runs on Microsoft Windows.
	
	\item uses the Ruby programming language.
	
	\item allows its users to create their own role-playing game with powerful dedicated tools.
\end{itemize}

This is only one instance among the long line of RPG Maker releases, some of which were released on other platforms, including consoles. Probably due to its early popularity, new projects continued to use this version way after it was superseded by newer ones.


%\subsubsection{Main features}

%TODO

\subsubsection{Ruby Game Scripting System}

\href{https://rmvxace.fandom.com/wiki/RGSS}{An article} writes : 

\say{RGSS stands for Ruby Game Scripting System, a library that has been used in RPG Maker engines since RPG Maker XP.  It's a Ruby library, and as such, all scripting is done in the Ruby language.
	
In all of its incarnations, RGSS has included objects and methods with which to handle graphics, audio, and data, including basic data structures for the RPG Maker engine.

The original version, RGSS, [...] was based on Ruby 1.81.}

RGSS's latest version, RGSS3, was introduced with RPG Maker VX Ace in 2011-2012.






\subsection{The Pokemon Essentials project}


According to its \href{https://essentialsdocs.fandom.com/wiki/Essentials_Docs_Wiki}{wiki fandom page} and \href{https://pokemon-fan-game.fandom.com/wiki/Pok\%C3\%A9mon_Essentials}{another wiki article}, Pokemon Essentials : 

\begin{itemize}
	\item \say{a collection of gameplay-altering original code designed for use in an RPG Maker XP game}.
	\item is a RMXP project, made to be the basis for a game.
	\item was forked from Flameguru's Pokémon Starter Kit.
	\item was developped by Peter O. (2007-2010) and Maruno (2011-today)
	\item was last updated in October of 2017, when it was hit by DMCA takedown notice.
	\item is the source of a fair proportion of Pokemon fan-made games.
	\item is up-to-date with the \nth{5} generation of the mainline Pokemon games.
\end{itemize}


\subsubsection{Digging for information on PE}

Here are some of my findings :

\begin{itemize}
\item Users have been able to produce content-adding patches, like support for newer generation creatures, items and abilities.

\item There is \href{https://www.reddit.com/r/PokemonRMXP/comments/ckeaov/pok\%C3\%A9mon\_essentials\_v18\_progress\_report/}{a v18 in the works} (release date unclear). \href{https://www.reddit.com/r/PokemonRMXP/comments/hb6i6m/should\_i\_wait\_for\_essentials\_v18/}{Here} Maruno stated "hopefully it won't be long now" in June 2020.

\item It is notably hosted  \href{https://archive.org/details/PokmonEssentialsV17.220171015}{on archive.org} but I downloaded it from \href{https://www.youtube.com/watch?v=-aLAoeZnRDw}{this source} because it has a Wikia dump and later file modification timestamps. %TODO:verify diffs
\end{itemize}






\subsubsection{Comments}

%TODO:eval if it's the right place for that

Programming with RPG Maker XP and Pokemon Essentials is complex, and is typically done in very small teams of amateurs, not using collaborative source code management tools like \textit{git}. This results in ambitious project requiring years before they can be released and are hard to maintain.

I believe this is the reason most projects never see a release, and the ones that do are buggy.



\subsection{The Pokemon Uranium project}


According to its \href{https://en.wikipedia.org/wiki/Pok\%C3\%A9mon_Uranium}{Wikipedia page}, its \href{http://pokemonuranium.org/}{official website} and its \href{https://pokemon-uranium.fandom.com/wiki/Main_Page}{Fan-made Wiki website}, Pokemon Uranium :
\begin{itemize}
	\item is a fan-made game
	\item uses the RPG Maker XP engine and Pokemon Essentials
	\item has unique region, creatures, plot and quests
	\item was developped by a small team (2 main devs) for about 9 years and released in August 2016.
	\begin{itemize}
		\item Game designer and developer : JV12345 (aka. $\sim$JV$\sim$)
		\item Game developer and creative director : Nageki (aka. Involuntary Twitch)
	\end{itemize}
	\item The development team was hit by DMCA takedown notices shortly after release. They stopped development in September 2016.
	\item Fan-made website and following game updates are community efforts.
\end{itemize}

\subsubsection{Digging for information on PU}

Here are some of my findings :

\begin{itemize}
	\item Fans have been able to patch bugs and update the game for some time. I suppose there is a small group of people with access to the project's source code and files that have been maintaining it and its online services.
	
	\item The game was mostly complete when released. Fans were able to add elements to push it farther towards completion but there still are some hanging threads, particularly in the post game (after the main plot is over).
	
	\item I wasn't able to identify the used versions of RPG Maker XP or Pokemon Essentials, but I determined this wouldn't significantly impact on working on it.
	
	\item Despite efforts to patch it, the game has a \href{https://pokemon-uranium.fandom.com/wiki/Bugs_and_Errors}{long list} of unresolved bugs and game breaking/crashing situations.
	
	\item GitHub user \textit{acedogblast} launched a \href{https://github.com/acedogblast/Project-Uranium-Godot}{re-implementation project} on the Godot game engine in early 2019. As of writing, he states that :
	\say{Only a limited portion of the game is playable currently.}
	
	\item Download links are hosted \href{https://www.reddit.com/r/pokemonuranium/comments/a0cw0i/download_links/}{on Reddit}. Current version is \verb|1.2.4| and was released on October \nth{29} 2018.
\end{itemize}


\newpage
\section{Research}

\subsection{Methodology}

Here are the high-level steps I took to study and experiment with RMXP, PE and PU :
\begin{enumerate}
	\item Installing a working copy of RMXP

	\item Opening the PE project
	
	This helped me understand how the project is structured, where the assets are located and how the scripting language works. Most of my experimentation started from it.
	
	\item Opening the PU project
	
	This helped me understand how developers implemented new features and edited PE to suit their needs.
\end{enumerate}

The objective was to understand how PE and its derivative works function and how features are implemented. I also needed to identify issues and come up with solutions to them, determine the specifics of the software I would implement in the next step of this project and document my findings.




\subsection{Technical findings}

\subsubsection{Structure}

PE and PU have a similar structure :

\begin{tabular}{|r L{.1\linewidth}|L{.8\linewidth}|}
	\hline
	$\circ$ & Root & Contains compiled game files and libraries, plus a few useful programs, some probably made by Pokemon Essentials developers. \\
	\hline
	\rotatebox[origin=c]{180}{$\Lsh$} & Audio & Contains the musical assets of the game, typically MIDI and OGG files. \\
	\hline
	\rotatebox[origin=c]{180}{$\Lsh$} & Data & Contains the logic of the game and most of its data.  \\
	\hline
	\rotatebox[origin=c]{180}{$\Lsh$} & Fonts & Contain fonts used for displaying text. \\
	\hline
	\rotatebox[origin=c]{180}{$\Lsh$} & Graphics & Contains the graphical assets of the game, typically tile sets, in a well organized folder structure. \\
	\hline
	\rotatebox[origin=c]{180}{$\Lsh$} & (\textit{PBS}) & (Facultative) Contains human-readable data to be compiled, like items, species, types, dialogue (and translations), etc. \\
	\hline
\end{tabular}




\subsubsection{Data representation}

The data that constitutes a PE project is diverse :

\begin{tabular}{|L{.1\textwidth}|L{.52\textwidth}|L{.3\textwidth}|}
	\hline
	\rowcolor{mylightgray}
	\textbf{Category} & \textbf{Description} & \textbf{Storage} \\
	\hline
	Scripts & Logic of the game, classes, ect & \verb|Scripts.rxdata| file \\
	\hline
	Objects & Files containing sets of class instances, grouped per file & \verb|dat| and \verb|rxdata| files \\
	\hline
	Maps & Contain map representation (incl. events), one per file & \verb|MapXXX.rxdata| \\
	\hline
	Dialogue & Contain dialogue & TODO \\
	\hline
\end{tabular}

Findings :
\begin{itemize}
	\item Data representation isn't human-readable. I found \verb|rxdata| and \verb|dat| files are either marshalled data or structured binaries. These can contain RGSS code or serialized objects or lists of objects.
	
	\item All \verb|rxdata| and most \verb|dat| files begin by the same two bytes, which is probably the magic number/signature for RPG Maker XP files : $\BS x04 \ \BS x08$
	
	Note that usually, signatures are 4 bytes long.
	
	\item I haven't been able to find any built-in way to export data, instead resorting to external tools. I have tried a few tools found on github or forums. %TODO:subsection about that
	
	This means I will either have to reverse-engineer them and build my own tool (probably in its native language Ruby to take advantage of the module/class definitions I may be able to find in the code) or find a way to code my own extraction scripts within RMXP and execute them.
\end{itemize}



\subsection{Non-technical findings}

Here are some of my research findings that doesn't fit in the previous section. 

\subsubsection{About fanmade games}

The projects I researched have a particular characteristic that I wasn't execting, but makes sense : They are personal side-projects from (mostly) individual developers. 

With this in mind, it is no surprise that they are very long running projects that took years to release and displayed other properties I used to find odd :
\begin{itemize}
	\item Closed source projects : Because they were very personal.
	\item Don't use code source management : Because there's no need, they are 1-2 developer personal projects that happened to be eventually released. Perhaps developers didn't consider the advantages of using one, or one wasn't available or familiar to them.
	\item Buggy projects (years after release) : With so few maintainers and such code complexity, patching bugs is a challenge.
\end{itemize} 

As if the challenges facing such endeavors weren't great enough already, they were struck with the threat of legal repercussions on their authors, forcing these to abandon their creation. I can't help but having sincere sentiments of admiration for their work and sadness for the tragic destiny of their projects.



Fortunately, we can learn from the failures of the past how to better go forward. It is my firm belief that there is a solution that solves the challenges that brought these projects down.

\newpage
\subsection{Extracting data}

In order to decouple PE from its original code base and RGSS, it became obvious that all useful data had to be extracted from the project.

Since \verb|Audio|, \verb|Graphics| and \verb|Fonts| directories contain already-accessible data, they didn't require any extraction efforts.

See section \textit{Data representation} for categories of data that were identified to need extraction.


\subsubsection{Scripts}

What I found :
\begin{itemize}
	\item \verb|Scripts.rxdata| is the second largest compiled file of the project, and contains over 120k lines of code.
	
	\item The script is unique : it is divided in named sections (that appear in RMXP's editor on the left hand side) that allow to manage its colossal line count.
	
	\item Like other compiled files, its content are almost impossible to read outside of RMXP.
\end{itemize}

I spent quite a lot of time working on a python script able to decrypt \verb|rxdata| files, and this one in particular, but didn't have much success, so I tried looking elsewhere.

\textbf{Trying something else} : After some more digging, I found a functional utility that was created for the purpose of editing/extracting the scripts from this file : Gemini Editor.

Here are links to \href{https://forum.chaos-project.com/index.php/topic,10420.0.html}{a forum post about it} and \href{https://github.com/terabin/Gemini}{a github repository}.

With it, I was able to extract Pokemon Essentials scripts (see '\verb|PE_scripts|' directory).

It proved invaluable to be able to explore/search the whole script base using a more powerful code editor, saving lots of time from my attempts to create my own scripts to export data.


\subsubsection{Maps}

What I found :
\begin{itemize}
	\item A tileset 
\end{itemize}










\newpage
\section{Remarks}

\subsection{Contact}

Contact the author by email : \href{mailto:David.Rodriguez.1@etu.unige.ch}{David.Rodriguez.1@etu.unige.ch}

\subsection{Privacy Policy}
\label{sec:privacypolice}

This document and its content are private and confidential. It is only intended for its academic recipient. It is strictly prohibited to copy, print, publish, share or distribute any part of it without written permission from its original author.

If you received this document by mistake, please inform its author and delete it. Thank you for your cooperation and understanding.















\end{document}
