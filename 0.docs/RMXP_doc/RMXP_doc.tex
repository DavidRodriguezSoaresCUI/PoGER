\documentclass[11pt]{article}

\usepackage{amssymb} % see: http://milde.users.sourceforge.net/LUCR/Math/mathpackages/amssymb-symbols.pdf
\usepackage{mathtools} % for extended mathematical symbols and more

% Page properties :
% See : https://tex.stackexchange.com/questions/36085/latex-without-pages
\usepackage{geometry}
\geometry{margin=16mm}


% fix for underlined text not hyphenating :
%\usepackage{soul} % use this
% then replace '\underline{}' with '\ul{}'

% To color text :
% see : https://en.wikibooks.org/wiki/LaTeX/Colors
\usepackage[dvipsnames]{xcolor} % use this
% then use one of these syntaxes :
% >\textcolor{declared-color}{text}
% >{\color{declared-color} text}

% Input file encoding :
\usepackage[utf8]{inputenc}     % makes accents not shit

% To use images : 
\usepackage{graphicx} % use this
\graphicspath{ {./IMG/} } % path midifier for storing images
% to include a picture, type this ('name' is without path or extention) :
% \includegraphics[width=0.5\textwidth]{name}

% Box : % width should be manually set
% \fbox{ \parbox{0.8\textwidth}{ text } }

% Espacement :
% \hfill \vfill
% \vspace{length} \hspace{length} % if ignored, use starred version or { \vspace*{length} }
% \phantom{text} 
% Indentation explicite :
% \indent et \noindent
\usepackage{parskip}

% Fractions :
% \frac{num}{denum}  % small text num/denum
% \dfrac{num}{denum} % normal text num/denum

% Coloring text: % works on text and math modes
% \textcolor{declared-color}{text}
% {\color{declared-color}some text}

% How to use references :
\usepackage{hyperref} % use this
% use '\hypertarget{some_label}{}', then
% and reference it with '\hyperlink{some_label}{some text}'
\hypersetup{
	colorlinks=true,    
	urlcolor=blue,
}

% To write prooftrees :
% see : http://www.pirbot.com/mirrors/ctan/macros/latex/contrib/ebproof/ebproof.pdf
%\usepackage{ebproof} % use this


% Verbatim blocks :
\usepackage{verbatim} % normal verbatim
\usepackage{fancyvrb} % for fancy boxed varbatims
\usepackage{parskip}

% For code
\usepackage{listings}
%\begin{lstlisting}
%Put your code he<re.
%\end{lstlisting}


\usepackage{color}

\definecolor{mylightgray}{rgb}{0.95,0.95,0.95}

\usepackage[super]{nth}

% Regulates \tableofcontents max depth
\setcounter{tocdepth}{2}

\usepackage{array}
\newcolumntype{L}[1]{>{\raggedright\let\newline\\\arraybackslash\hspace{0pt}}m{#1}}
\newcolumntype{C}[1]{>{\centering\let\newline\\\arraybackslash\hspace{0pt}}m{#1}}
\newcolumntype{R}[1]{>{\raggedleft\let\newline\\\arraybackslash\hspace{0pt}}m{#1}}

\usepackage{xcolor,colortbl}
\newcolumntype{a}{>{\columncolor{mylightgray}}l}

\usepackage{array}
\setlength\extrarowheight{4pt} % or whatever amount is appropriate

\lstset{ 
	backgroundcolor=\color{mylightgray},   % choose the background color; you must add \usepackage{color} or \usepackage{xcolor}; should come as last argument
	basicstyle=\small,        % the size of the fonts that are used for the code
	breakatwhitespace=false,         % sets if automatic breaks should only happen at whitespace
	breaklines=true,                 % sets automatic line breaking
	captionpos=b,                    % sets the caption-position to bottom
	commentstyle=\color{mygreen},    % comment style
	deletekeywords={...},            % if you want to delete keywords from the given language
	escapeinside={\%*}{*)},          % if you want to add LaTeX within your code
	extendedchars=true,              % lets you use non-ASCII characters; for 8-bits encodings only, does not work with UTF-8
	firstnumber=1,                % start line enumeration with line 1000
	frame=single,	                   % adds a frame around the code
	keepspaces=true,                 % keeps spaces in text, useful for keeping indentation of code (possibly needs columns=flexible)
	keywordstyle=\color{blue},       % keyword style
	%language=Java,                 % the language of the code
	morekeywords={},            % if you want to add more keywords to the set
	numbers=none,                    % where to put the line-numbers; possible values are (none, left, right)
	numbersep=5pt,                   % how far the line-numbers are from the code
	numberstyle=\tiny\color{mygray}, % the style that is used for the line-numbers
	rulecolor=\color{black},         % if not set, the frame-color may be changed on line-breaks within not-black text (e.g. comments (green here))
	showspaces=false,                % show spaces everywhere adding particular underscores; it overrides 'showstringspaces'
	showstringspaces=false,          % underline spaces within strings only
	showtabs=false,                  % show tabs within strings adding particular underscores
	stepnumber=2,                    % the step between two line-numbers. If it's 1, each line will be numbered
	stringstyle=\color{mygreen},     % string literal style
	tabsize=2,	                   % sets default tabsize to 2 spaces
	title=\lstname                   % show the filename of files included with \lstinputlisting; also try caption instead of title
}
\usepackage[english]{babel}
\usepackage{lipsum}
\usepackage{multirow,booktabs} % https://jdhao.github.io/2019/08/27/latex_table_with_booktabs/

\newcounter{codeID}[section]
\newcommand{\nextcodeID}{\refstepcounter{codeID}\multirow{4}{*}{\thecodeID}}

\usepackage{longtable}

\newcommand{\degrees}[1]{#1$^{\circ}$}


%opening
\title{RPG Maker XP documentation}
\author{David Rodriguez Soares}


% first page
\begin{document}
\maketitle

\vspace{90mm}

\textbf{Privacy policy}

This is a confidential document and should not be distributed under any circumstance. \hyperref[sec:privacypolice]{Please click and read}.

\textbf{Abstract}

\lipsum[2-3]


\newpage

\begingroup
\hypersetup{linkcolor=black}
\tableofcontents
\endgroup

\newpage
\section{What this document is about}

%TODO


\newpage
\section{Events}

An event, or more precisely a \textit{map event}, is a way to introduce elements with behavior, therefore bringing flexibility and dynamism into the game world.

Events have two aspects :
\begin{itemize}
	\item A GUI element.
	
	\begin{center}
		\includegraphics[width=0.54\textwidth]{Event} 
	\end{center}

	\item Its data class instance counterpart.
	
	\begin{center}
		\includegraphics[width=0.85\textwidth]{Event_json}
		
	\end{center}
\end{itemize}

Basic information can be found in RPG Maker XP's provided documentation.

\newpage
\subsection{Basic functionalities}

These are the easiest and most straightforward behavior to implement into an event :

\begin{itemize}
	\item Giving an element a \textit{sprite} (texture) : This is useful for objects capable of movement, NPCs, etc.
	
	\item \textit{Movement} : Select how the element moves with presets (speed, frequency, pattern, etc).
	
	\item \textit{Event commands} : Select the trigger for behavior and what the element does when triggered (movement, dialogue, etc) within the extensive command list.
\end{itemize}

\subsection{Advanced functionalities}

These require an understanding of conditional execution and scripting :

\begin{itemize}
	\item \textit{Conditional execution} : branching instructions based on the value of : global variables, global switches, self switches, script return, etc.
	
	\item \textit{Pages} : Allow to give an element different behavior depending on conditions.
	
	\item \textit{Move routes} : Define a sequence of movement commands to be executed.
	
	\item \textit{Script calls} : Call a script to be executed for more complex behavior, launching mini-games, retrieving data, etc.
\end{itemize}



\newpage
\section{Commands}

Although they are very similar in structure and use, a distinction is made between \verb|RPG::EventCommand| and \verb|RPG::MoveCommand|.

\textit{EventCommands} are the representation of elements present in the "List of Event Commands" in the GUI. They are the building block of event's behavior.

\textit{MoveCommands} are the representation of an individual movement the event is capable of, typically found in sequences \verb|RPG::MoveRoute| associated with a dedicated \textit{EventCommand}.

They both have, at least :
\begin{itemize}
	\item A \textit{code} : An integer that uniquely identifies the particular command.
	\item \textit{Parameters} : Depend on the particular command, can be empty, a variable, an object, or a list of objects.
\end{itemize}

Additionally, \textit{EventCommands} have an \textit{indent} integer value, tied to the layout visible in the "List of Event Commands" in the GUI.


\subsection{Methodology}

In order to successfully \textit{extract semantic from events}, it was decided that \textit{documenting} every command used in Pokemon Essentials and finding an appropriate (human-readable) \textit{representation} was the way forward.

% I tested a few table designs

%\begin{table}[h!]
%	\begin{center}
%		\caption{More columns.}
%		\label{tab:table1}
%		\begin{tabular}{l|c|r|l}
%			\textbf{Value 1} & \textbf{Value 2} & \textbf{Value 3} & \textbf{Value 4}\\ % <-- added & and content for each column
%			$\alpha$ & $\beta$ & $\gamma$ & $\delta$ \\ % <--
%			\hline
%			1 & 1110.1 & a & e\\ % <--
%			2 & 10.1 & b & f\\ % <--
%			3 & 23.113231 & c & g\\ % <--
%		\end{tabular}
%	\end{center}
%\end{table}

%\begin{tabular}{c l l}
%	\toprule
%	\multirow{3}{*}[-2mm]{1} & Description & \verb|RPG::MoveCommand| - "Move Down" \\
%	\cmidrule{2-3}
%	& Parameters & None \\
%	\cmidrule{2-3}
%	& Representation & "Move Down" \\ [1mm]
%	\toprule
%\end{tabular}

%\begin{tabular}{c a l}
%	\toprule
%	\multirow{3}{*}[-2mm]{1} & Description & \verb|RPG::MoveCommand| - "Move Down" \\
%	\cmidrule{2-3}
%	& Parameters & None \\
%	\cmidrule{2-3}
%	& Representation & "Move Down" \\ [1mm]
%	\toprule
%\end{tabular}

\subsection{Miscellaneous information}

Codes used in Pokemon Essentials 17.2 :
\begin{quote}
	0, 1, 2, 3, 4, 5, 6, 7, 8, 9, 10, 11, 12, 13, 14, 15, 16, 17, 18, 19, 20, 21, 22, 23, 24, 25, 26, 33, 34, 37, 38, 39, 40, 41, 42, 44, 101, 102, 104, 106, 108, 111, 112, 113, 115, 118, 119, 121, 122, 123, 125, 201, 202, 208, 209, 210, 221, 222, 223, 225, 231, 232, 235, 236, 241, 242, 247, 248, 249, 250, 314, 354, 355, 401, 402, 404, 408, 411, 412, 413, 655
\end{quote}

Implementation details :
\begin{itemize}
	\item \verb|RPG::MoveCommand| use range [1-45]
	
	\item \verb|RPG::EventCommand| use range [101-$x$], $x\geq 655$
	
	\item A "frame" is defined as $\dfrac{1}{20}$ second.
\end{itemize}


\newpage
\setcounter{codeID}{-1}
{\small
\begin{tabular}{|c a l|}
	\hline
	\nextcodeID & Description & Nothing, empty command or end of the event command list \\
	& Parameters & None \\
	& Notes & Will not be represented \\
	& Representation & None \\
	\hline
	\nextcodeID & Description & \verb|RPG::MoveCommand| - Move to the South \\
	& Parameters & None \\
	& Notes & See footnote\footnotemark[1] \\
	& Representation & "Move, S" \\
	\hline
	\nextcodeID & Description & \verb|RPG::MoveCommand| - Move to the West \\
	& Parameters & None \\
	& Notes & See footnote\footnotemark[1] \\
	& Representation & "Move, W" \\
	\hline
	\nextcodeID & Description & \verb|RPG::MoveCommand| - Move to the East \\
	& Parameters & None \\
	& Notes & See footnote\footnotemark[1] \\
	& Representation & "Move, E" \\
	\hline
	\nextcodeID & Description & \verb|RPG::MoveCommand| - Move to the North \\
	& Parameters & None \\
	& Notes & See footnote\footnotemark[1] \\
	& Representation & "Move, N" \\
	\hline
	\nextcodeID & Description & \verb|RPG::MoveCommand| - Move to the SouthWest \\
	& Parameters & None \\
	& Notes & See footnote\footnotemark[1] \\
	& Representation & "Move, SW" \\
	\hline
	\nextcodeID & Description & \verb|RPG::MoveCommand| - Move to the SouthEast \\
	& Parameters & None \\
	& Notes & See footnote\footnotemark[1] \\
	& Representation & "Move, SE" \\
	\hline
	\nextcodeID & Description & \verb|RPG::MoveCommand| - Move to the NorthWest \\
	& Parameters & None \\
	& Notes & See footnote\footnotemark[1] \\
	& Representation & "Move, NW" \\
	\hline
	\nextcodeID & Description & \verb|RPG::MoveCommand| - Move to the NorthEast \\
	& Parameters & None \\
	& Notes & See footnote\footnotemark[1] \\
	& Representation & "Move, NE" \\
	\hline
	\nextcodeID & Description & \verb|RPG::MoveCommand| - Move at random (N,E,S,W) \\
	& Parameters & None \\
	& Notes & See footnote\footnotemark[1] \\
	& Representation & "Move, R" \\
	\hline
	\nextcodeID & Description & \verb|RPG::MoveCommand| - Move towards player \\
	& Parameters & None \\
	& Notes & See footnotes\footnotemark[1]\textsuperscript{,}\footnotemark[3] \\
	& Representation & "Move, TODO" \\
	\hline
\end{tabular}

\newpage
\begin{tabular}{|c a l|}
	\hline
	\nextcodeID & Description & \verb|RPG::MoveCommand| - Move away from player \\
	& Parameters & None \\
	& Notes & See footnotes\footnotemark[1]\textsuperscript{,}\footnotemark[3] \\
	& Representation & "Move, TODO" \\
	\hline
	\nextcodeID & Description & \verb|RPG::MoveCommand| - Take 1 step forward \\
	& Parameters & None \\
	& Notes & See footnote\footnotemark[1] \\
	& Representation & "Move, TODO" \\
	\hline
	\nextcodeID & Description & \verb|RPG::MoveCommand| - Take 1 step backward \\
	& Parameters & None \\
	& Notes & See footnote\footnotemark[1] \\
	& Representation & "Move, TODO" \\
	\hline
	\nextcodeID & Description & \verb|RPG::MoveCommand| - Jump to relative coordinates on the same map \\
	& Parameters & [2] - \textbf{0}:deltaX \verb|[signed integer]|, \ \textbf{1}:deltaY \verb|[signed integer]| \\
	& Notes &  \\
	& Representation & "Jump, TODO" \\
	\hline
	\nextcodeID & Description & \verb|RPG::MoveCommand| - Wait n seconds \\
	& Parameters & [1] - \textbf{0}:number of seconds to wait $n$ \verb|[integer|$\;\in \mathbb{N}^*$\verb|]| \\
	& Notes & Typically $n==2$, but values up to 15 were found in PE. \\
	& Representation & "Wait seconds, $n$" \\
	\hline
	\nextcodeID & Description & \verb|RPG::MoveCommand| - Turn towards South \\
	& Parameters & None \\
	& Notes & See footnote\footnotemark[2] \\
	& Representation & "Turn, S" \\
	\hline
	\nextcodeID & Description & \verb|RPG::MoveCommand| - Turn towards West \\
	& Parameters & None \\
	& Notes & See footnote\footnotemark[2] \\
	& Representation & "Turn, W" \\
	\hline
	\nextcodeID & Description & \verb|RPG::MoveCommand| - Turn towards East \\
	& Parameters & None \\
	& Notes & See footnote\footnotemark[2] \\
	& Representation & "Turn, E" \\
	\hline
	\nextcodeID & Description & \verb|RPG::MoveCommand| - Turn towards North \\
	& Parameters & None \\
	& Notes & See footnote\footnotemark[2] \\
	& Representation & "Turn, N" \\
	\hline
	\nextcodeID & Description & \verb|RPG::MoveCommand| - Turn  \degrees{90} right, relative to current position \\
	& Parameters & None \\
	& Notes & See footnote\footnotemark[2] \\
	& Representation & "Turn, R" \\
	\hline
	\nextcodeID & Description & \verb|RPG::MoveCommand| - Turn  \degrees{90} left, relative to current position \\
	& Parameters & None \\
	& Notes & See footnote\footnotemark[2] \\
	& Representation & "Turn, L" \\
	\hline
\end{tabular}

\newpage
\begin{tabular}{|c a l|}
	\hline
	\nextcodeID & Description & \verb|RPG::MoveCommand| - Turn  \degrees{180} \\
	& Parameters & None \\
	& Notes & See footnote\footnotemark[2] \\
	& Representation & "Turn, 180" \\
	\hline
	\nextcodeID & Description & \verb|RPG::MoveCommand| - Turn  \degrees{90} to the left or right, at random \\
	& Parameters & None \\
	& Notes & See footnote\footnotemark[2] \\
	& Representation & "Turn, 90random" \\
	\hline
	\nextcodeID & Description & \verb|RPG::MoveCommand| - Turn  at random (\degrees{90} or \degrees{180}) \\
	& Parameters & None \\
	& Notes & See footnote\footnotemark[2] \\
	& Representation & "Turn, random" \\
	\hline
	\nextcodeID & Description & \verb|RPG::MoveCommand| - Turn towards player \\
	& Parameters & None \\
	& Notes & See footnotes\footnotemark[2]\textsuperscript{,}\footnotemark[3] \\
	& Representation & "Turn, TODO" \\
	\hline
	\nextcodeID & Description & \verb|RPG::MoveCommand| - Turn away from player \\
	& Parameters & None \\
	& Notes & See footnotes\footnotemark[2]\textsuperscript{,}\footnotemark[3] \\
	& Representation & "Turn, TODO" \\
	\hline
	\setcounter{codeID}{32}
	\nextcodeID & Description & \verb|RPG::MoveCommand| - Turn ON walking animation \\
	& Parameters & None \\
	& Notes &  \\
	& Representation & "Animation, ON" \\
	\hline
	\nextcodeID & Description & \verb|RPG::MoveCommand| - Turn OFF walking animation \\
	& Parameters & None \\
	& Notes &  \\
	& Representation & "Animation, OFF" \\
	\hline
	\setcounter{codeID}{36}
	\nextcodeID & Description & \verb|RPG::MoveCommand| - Turn ON "through" \\
	& Parameters & None \\
	& Notes & \parbox{.7\linewidth}{Equivalent to activating "walk through walls", making it possible to walk through impassable tiles/characters.} \\
	& Representation & "WTW, ON" \\
	\hline
	\nextcodeID & Description & \verb|RPG::MoveCommand| - Turn OFF "through" \\
	& Parameters & None \\
	& Notes & Equivalent to deactivating "walk through walls". \\
	& Representation & "WTW, OFF" \\
	\hline
	\nextcodeID & Description & \verb|RPG::MoveCommand| - Always on top ON \\
	& Parameters & None \\
	& Notes & \parbox{.7\linewidth}{Elevate the display priority, therefore bringing the event graphic to the forefront (above any tile/character)} \\
	& Representation & "AOT, ON" \\
	\hline
	\nextcodeID & Description & \verb|RPG::MoveCommand| - Always on top OFF \\
	& Parameters & None \\
	& Notes &  \\
	& Representation & "AOT, OFF" \\
	\hline
\end{tabular}

\newpage
\begin{tabular}{|c a l|}
	\hline
	\nextcodeID & Description & \verb|RPG::MoveCommand| - Change event's graphic \\
	& Parameters & TODO \\
	& Notes &  \\
	& Representation & "TODO" \\
	\hline
	\nextcodeID & Description & \verb|RPG::MoveCommand| - Change event's graphic opacity \\
	& Parameters & [1] - \textbf{0}:new opacity value $n$ \verb|[integer 0-255]| \\
	& Notes &  \\
	& Representation & "Opacity, $n$" \\
	\hline
	\setcounter{codeID}{43}
	\nextcodeID & Description & \verb|RPG::MoveCommand| - Play a sound effect \\
	& Parameters & TODO \\ %[1] - \textbf{0}:sound effect \verb|[RPG::AudioFile]| \\
	& Notes &  \\
	& Representation & "Play SE, TODO" \\
	\hline
	\multirow{4}{*}[-1mm]{101} & Description & \verb|RPG::EventCommand| - Show text \\
	& Parameters & [1] - \textbf{0}:text $s$ \verb|[String]| \\
	& Notes & \parbox{.7\linewidth}{$s$ must be properly double-quoted and formatted (inner double-quotes and backslashes must be escaped).} \\
	& Representation & "Show Text, $s$" \\
	\hline
	\multirow{4}{*}{401} & Description & \verb|RPG::EventCommand| - Show text (continued) \\
	& Parameters & [1] - \textbf{0}:text $s$ \verb|[String]| \\
	& Notes & Continuation of 101. \\
	& Representation & See footnote\footnotemark[4] \\
	\hline
	\multirow{4}{*}{102} & Description & \verb|RPG::EventCommand| - Show choices \\
	& Parameters & [2] - \textbf{0}:array of size $n$ \verb|[Array of Strings]|, \ \textbf{1}:cancel behaviour \verb|[integer 0-4]| \\
	& Notes & \parbox{.7\linewidth}{Displays up to 4 selectable options in a message window. Cancel behaviour : 0 disallow canceling, 1-4$\leq n$ selects choice by default.} \\
	& Representation & "Choose, \{0\}, default=\{1\}" \\
	\hline
	\multirow{4}{*}[-1mm]{104} & Description & \verb|RPG::EventCommand| - Change text options \\
	& Parameters & [2] - \textbf{0}:position $p$ \verb|[integer 0-2]|, \ \textbf{1}:window border $b$ \verb|[integer 0-1]| \\
	& Notes & \parbox{.7\linewidth}{Sets message window position and border. $p$ follows "common relation 1", $b$ follows "common relation 2"} \\
	& Representation & "Change text options, position=\{0\}.toString(), border=\{1\}.toString()" \\
	\hline
	\multirow{4}{*}[-1mm]{106} & Description & \verb|RPG::EventCommand| - Wait \\
	& Parameters & [1] - \textbf{0}:number of frames to wait $n$ \verb|[integer|$\;\in \mathbb{N}^*$\verb|]| \\
	& Notes & \parbox{.7\linewidth}{Conversion to milliseconds chosen for its more precise and general use : $m=n*1000/20\equiv n*50$, TODO:research its use}  \\
	& Representation & "Wait ms, $m$" \\
	\hline
	\multirow{4}{*}{108} & Description & \verb|RPG::EventCommand| - Comment \\
	& Parameters & [1] - \textbf{0}:comment text $s$ \verb|[String]| \\
	& Notes & Has no effect. TODO:research link to particle effects. \\
	& Representation & "\# $s$" \\
	\hline
	\multirow{4}{*}{408} & Description & \verb|RPG::EventCommand| - Comment (continued) \\
	& Parameters & [1] - \textbf{0}:comment text $s$ \verb|[String]| \\
	& Notes & Happens after a 108. \\
	& Representation & "\# $s$" \\
	\hline
\end{tabular}

\newpage
\begin{tabular}{|c a l|}
	\hline
	\multirow{4}{*}{111} & Description & \verb|RPG::EventCommand| - Conditional branch \\
	& Parameters & See \hyperref[sec:condbranch]{"Conditional branch" section}. \\
	& Notes & Complex but essential command. \\
	& Representation & "If, \{condition\}" \\
	\hline
	\multirow{4}{*}{112} & Description & \verb|RPG::EventCommand| - Loop \\
	& Parameters & None \\
	& Notes & Loops over commands until broken. TODO:research usage \\
	& Representation & "Loop" \\
	\hline
	\multirow{4}{*}{113} & Description & \verb|RPG::EventCommand| - Break loop \\
	& Parameters & None \\
	& Notes & Escape innermost loop. TODO:research usage \\
	& Representation & "Break" \\
	\hline
	\multirow{4}{*}{115} & Description & \verb|RPG::EventCommand| - Exit Event Processing \\
	& Parameters & None \\
	& Notes & TODO:research usage \\
	& Representation & TODO \\
	\hline
	\multirow{4}{*}{118} & Description & \verb|RPG::EventCommand| - Label \\
	& Parameters & [1] - \textbf{0}:label name $s$ \verb|[String]| \\
	& Notes & Sets a label to allow jumping to. \\
	& Representation & "Label, $s$" \\
	\hline
	\multirow{4}{*}{119} & Description & \verb|RPG::EventCommand| - Jump to Label \\
	& Parameters & [1] - \textbf{0}:label name $s$ \verb|[String]| \\
	& Notes & Jumps to a label. \\
	& Representation & "Jump to Label, $s$" \\
	\hline
	\multirow{4}{*}[-1mm]{121} & Description & \verb|RPG::EventCommand| - Control switches \\
	& Parameters & [3] - \textbf{0}:starting switch $ssa$ \verb|[integer]|, \ \textbf{0}:starting switch $ssz$ \verb|[integer]|, \ \textbf{0}:new state $n$ \verb|[integer]| \\
	& Notes & \parbox{.7\linewidth}{Batch control is unused in PE, therefore deprecated. $n$ follows "common relation 3".} \\
	& Representation & "Control Switch, $ssa$.toString(), $n$.toString()" \\
	\hline
	\multirow{4}{*}[-1mm]{121} & Description & \verb|RPG::EventCommand| - Control variables \\
	& Parameters & See \hyperref[sec:varctrl]{"Control variables" section}. \\
	& Notes & \parbox{.7\linewidth}{Batch control is unused in PE, therefore deprecated. $n$ follows "common relation 3".} \\
	& Representation & "Control Switch, $ssa$.toString(), $n$.toString()" \\
	\hline
\end{tabular}}

\footnotetext[1]{Movements consolidated with new \textit{Move} command with argument.}
\footnotetext[2]{Turs consolidated with new \textit{Turn} command with argument.}
\footnotetext[3]{Unknown algorithm to determine direction "towards player" and "away from player.}
\footnotetext[4]{Is part of a command sequence that should be merged in a sensible way.}

Common relations :
\begin{enumerate}
	\item 0:Top, 1:Middle, 2:Bottom
	\item 0:Show, 1:Hide
	\item 0:ON, 1:OFF
\end{enumerate}


\subsection{Complex commands}

Some commands have complex behaviour that doesn't fit in the table above, so I put detailed explanation below


\subsubsection{Conditional branch}
\label{sec:condbranch}

\subsubsection{Control variables}
\label{sec:varctrl}



\newpage
\section{Remarks}

\subsection{Contact}

Contact the author by email : \href{mailto:David.Rodriguez.1@etu.unige.ch}{David.Rodriguez.1@etu.unige.ch}

\subsection{Privacy Policy}
\label{sec:privacypolice}

This document and its content are private and confidential. It is only intended for its academic recipient. It is strictly prohibited to copy, print, publish, share or distribute any part of it without written permission from its original author.

If you received this document by mistake, please inform its author and delete it. Thank you for your cooperation and understanding.




















\end{document}
